\section{Methodology} % < 1.5 Pages

% Outlines:
% - What gestures we want to recognise, based on papers from lit review (maybe own small section?)
% - Taking a data driven approach, what data will we collect

% Distinguishing Between Head and Phone Movement (Methodology)
%   Redefine goals of the system/research objectives
%   Rough outline of the technology we aim to use?
%   Outline data we need

This section details the process undertaken to develop the system which can meet the goal (outlined in \autoref{sec:intro}): distinguishing between head and phone based gestures on a smartphone.

% \subsection{Taking A Data Driven Approach}

\subsection{Data Collection Study} % < 1.5 Pages
% outline
% - Taking data driven approach (intro to section)
% - The data we'll be collecting
% - The tools and apparatus we'll be using
% - The data collection app (how it works, issues, output format)
% - Gestures we chose to collect (based on lit review systems, (only subset of pointing, only looking at 8 points around the screen, rather than anywhere))
% - Study Outline
%   - What participants would be asked to do
%   - Need footnote or aside to mention that originally intended to also capture IMU data from an eSense earbud, but was not ultimately collected, hence why earbud included in steps
% - The study data
%   - breakdown of participants
%   - Recordings captured (/motions)
%   - Data analysis (distance moved by gesture, time taken, face detected (using YuNet and OpenCV))
%       - Why might expect
%       - Issue accurately calculating rotation delta (maybe use this? https://forum.unity.com/threads/shortest-rotation-between-two-quaternions.812346/)


\subsubsection{Apparatus and Techniques}\nl % Different Header?
Given the data types listed above, we decided to use the following tools:
\begin{description}
    \item[Smartphone]\nl Pixel 4a
    An Android Smartphone with Bluetooth, a front-facing camera, and an IMU
    \item[eSense Earable] - A Bluetooth Earbud with an IMU
    \item[CAMERA Motion Capture Studio]\nl - A Motion Capture (MoCap) studio found on campus within the University of Bath
\end{description}
The smartphone and earbud (when paired via bluetooth) will be able to provide the first three data-types defined above. While the Ground-Truth data can then be supplied by the MoCap studio.
\\\\
To collect the data we developed an application to run on the smartphone. 
% Need this?
This was developed in Kotlin\footnote{A programming language that runs on the JVM and is used to develop applications for Android.} and the Android SDK.
The application was designed to show participants a motion (a gesture and direction/variation) to perform. This is detailed in text, images, and a video. \tempnote{Include figures to show this (spread accross columns at top of page?)}
The participant would then be asked to perform this motion after pressing a record button. While recording the app would do the following:
\begin{itemize}
    \item Capture images as frequently as possible from the front-facing camera, saving them as raw YUV bytes, with the UTC timestamp as the name.
    \item Record the smartphone IMU data (linear and angular accelerations), saving them to a csv with the UTC timestamp.
    \item Record the earbud IMU data (linear and angular accelerations), saving them to a csv with the UTC timestamp.
\end{itemize}
% Link to class diagram, or do we just want photos?
Once the participant has finished with the motion they could press the same button to stop the recording. Otherwise the recording will automatically terminate after 10 seconds, since the gestures shouldn't take more than a couple seconds to perform and the phone has limited RAM and storage with which to save data.
To prevent accidentally stopping the recording too soon, say by accidentally double-tapping the screen, we disable the button for 2 seconds.

Once a motion has been recorded, the app shows the participant the next motion to perform. When the participant completes the final motion to perform, the app returns to the first motion. This repeats two times, such that each motion is captured 3 times. This is to collect variance in each motion for each participant.
\\\\
In order to collect the Ground-Truth data, the study was performed within the MoCap studio. The participant was asked to wear hat that had a motion-tracker attached, such that the tracker was placed around the middle of the back of their head. An exact position wasn't important as we only needed to determine the relative movement of their head, rather than the exact position.
The smartphone was then tracked via a motion tracker attached to a 3D-printed mount, such that the tracker would not affect the participant's grip on the phone, or interfere with the images captured from the front-facing camera.\tempnote{Include figure to show this}
Each tracker was composed of 5 points. 3 were positioned such that they formed a right-angle triangle, allowing the orientation of the tracker to be derived. The other 2 points were there to improve tracking accuracy, and help make the trackers unique and distinguishable.
The MoCap system would track each of the 10 points at 60 fps and export the data as an fbx file.

In order to later synchronise the data collected we required the user to shake the smartphone, with a force of at least 2G, prior to beginning each round of 44 motions. The app would record the shake magnitude (in the X, Y, and Z axis) and the UTC timestamp of when it happened.
\\\\
The full study protocol can be found under \autoref{app:protocol}

\subsubsection{Study Results}
Our study was run with 8 participants.

Unfortunately due to an issue with the application, the earbud IMU data was not recorded, despite the earbud being on and paired with the phone. This was not caught until after the study was completed.

Due to a late start, and overrunning into the next participant's slot, participant 0 was unable to complete their 3\textsuperscript{rd} round of motions.

Some participants didn't initially stop recording upon completing a motion, as such their initial motions have superfluous frames that don't contain data relevant to the motion they're recorded for.

\tempnote{Stats from the data, including figs and tables? Range of motion per gesture. Time taken by gesture and participant, Average sample rate for IMU and Images, Average Number of frames containing face by participant and gesture}
% When Performed, Participants recruited, participant breakdown/stats
% Missing earable data!!

% Some participants didn't stop recording, contain misc frames
% Some stats on the data?



\subsection{Data Synchronisation and Post-Processing}
Before being able to use the data for training, we needed to synchronise the data recorded from the smartphone, and the fbx data from the MoCap studio.
To do this we derived the acceleration of the phone based on the MoCap data to find where it meets/exceeds the magnitude of the shake recorded by the app. From this we can determine the frame of the fbx data that corresponds to the recorded timestamp. We can determine the frame for any subsequent timestamp based on the known frame-rate of 60fps. 
To verify the data didn't drift we resync the data based on the other 2 recorded shakes, verifying that they're within 10 frames of the expected frame. In doing this verification, we did not come across any recording wherein subsequent shakes were not found to be at the expected frame.
The synchronisation was performed with a Python script that was run within Blender after loading in the fbx.
Blender was used as it permitted viewing the fbx data so that we could verify the frames the shake was detected by watching the playback. We could also use Blender to verify the derived location, roll, pitch, and yaw were correct by inserting a plane into the scene and assigning it the derived values, checking that it lined-up through all of the motion trackers.
The other reason for using Blender is that it allowed us to programmatically access the fbx file as a data structure and therefore write a script capable of deriving the required head and phone poses from the fbx, and then sync it with the data recorded from the phone.
Synchronised data was exported to CSVs for each motion recording, containing a path to the image, the raw IMU data, and the derived MoCap data.
\tempnote{Link to script used.}

An additional stage of pre-processing that was performed was to generate the RGB images from the YUV data captured by the phone. This was performed with a Python script and the OpenCV library.\tempnote{Link to script used.}
% script.processor.process_data('G:/Study/7', 1900)
% script.processor.process_data('G:/Study/6', 8900)
% script.processor.process_data('G:/Study/5', 9500)
% script.processor.process_data('G:/Study/4', 900)
% script.processor.process_data('G:/Study/3', 20000)
% script.processor.process_data('G:/Study/2', -320, attempt_0_override=-320)
% script.processor.process_data('G:/Study/1', 1000)
% script.processor.process_data('G:/Study/0', 3000)

\subsection{The Proposed Models} % TODO: Drop /s if only one model % The Proposed System
% Both models then used to train HMM, evaluate which performs better
% In this section we shall propose two models for quantising the motion of the phone and position of the head, and a HMM to classify sequences of the encoded motion as specific gestures.
In this section we shall propose a Neural-Network Cascade for quantising the position and pose of the head, and a HMM to classify sequences of the poses as a gesture.
% Possibly one-hot encoding
The purpose of splitting our proposed model into 2 distinct models (the neural network and HMM) is two-fold:

\nl\textbf{Obtaining the Face Pose}\nl
To be able to track the movement of the face, we first need to extract it and determine its pose.

\nl\textbf{Reduce the Observation Space For the HMM}\\
Since the acceleration can be any value, and the position of the face in the image also any range of values, to use them as raw inputs to a HMM would require a significant amount of data to ensure we have samples that cover the training space.
By first encoding the data we can reduce the possible training space.
The simplest way to perform this would be to quantise the data. This would involve reducing the resolution of the data, for example mapping all the angles of rotation into a smaller range, as was performed by \citeauthor{elmezain2008hidden} to convert the movement of a hand capture in a sequence of images to the angle of the movement\cite{elmezain2008hidden}, reducing the possible input to their HMM to just 19 observable states.

\subsubsection{Pose and Movement Quantisation}\nl
% Want to reduce the possible observation states we need to learn for our classification HMM.
% \nl\textbfit{Classifier 1}\nl
% 2: Cascading motion encoder
%   No transfer learning
%   use YuNet to preprocess the image. 
%   - If no face found then return no encoding for face
%   - Else feed into NN to encode the motion
%   Acceleration encoded manually, just quantise and see if over given amount, then convert to encoding
For this model we take inspiration from the cascading classifiers we reviewed\cite{kim2017real, neto2012real, francone2011using,viola2004robust}, wherein we will utilise the YuNet face detection CNN\cite{yu2022yunet} which is used with OpenCV to extract the bounding box of the face, and the positions of the eyes, nose, and mouth corners. To improve robustness of the classifier we will reduce the input image size down to 320x180, from the 1280*720 resolution we captured.

If no face is found we simply return a state representing no face detected, otherwise we will quantise the output by dividing the values by 10, and round to the nearest int. This will reduce the total possible values from 0-57,600 to 0-576.
Once the values have been quantised, they will be fed into a new model which we will train to predict the roll, pitch, and yaw of the face.
This is a small Fully Connected Network which will be trained on YuNet outputs, trying to predict a quantised roll, pitch, and yaw that we will generate from the MoCap data.
We do not need to try and predict the position of the head using the MoCap data as YuNet already gives use the position of the head in the frame. We will then return the quantised position of the face bounding box and landmarks, along with the quantised pose of the head predicted by by our model.
\tempnote{Graph of model, description of Layers}
Acceleration will be quantised in the same manner, multiplying by 5 and rounding to the nearest int. This should remove some noise and minor adjustments to the phone's position, and only track movements over 0.2ms\textsuperscript{-2}.

To build this model we will be using TensorFlow, with the Keras API, as TensorFlow provides as Mobile version of the network that can be run on smartphones.

% might encode one-hot instead, in which case take two sequential inputs and calculate the movement

% \nl\textbfit{Classifier 2}\nl
% % 2: NN using transfer learning (using image net model, which one?).
% %   Use tensorflow (work with current cuda and dependencies?) or pytorch?
% %   - 4 inputs, 2 images, 2 sets of acceleration data
% %     Images fed into own image net model, outputs into FCN and then both connected into FCN
% %     Acceleration fed into FCN, then joined with FCN of images
% %   Output one-hot encoded direction of motion for each DoF
% %   might be slow as re-processing previous image
% Our second proposed model we will be using transfer learning to extend an existing CNN trained on image net to predict the direction of motion, in each degree of freedom for both the head and phone, observed between two frames.
% We could try to emulate the YuNet model, and output the head pose and position, a lack of 
% One issue is lack of images not containing face, how to handle if no face detected?

\subsubsection{Gesture Classification}\nl
% Could have used an RNN bolted directly onto the back of the classifiers
Once we have our quantised observations from our cascade, we can now look to classifying the sequence of these observations as gestures.
For this we have chosen to train a HMM, as it will allows us to encode the gestures as the hidden state, with the quantised data as the observed states.

To build and train this we will first collect the output from out first model to generate the training data required. This will give us the data sequences we would expect to see in different gestures.
To build the HMM we will use the HMMLearn python package. Implementation of the model in this scenario shouldn't matter as much as our previous model as evaluating a HMM is much more trivial than a neural network. You should be able to export the state and variables of the HMM and transfer these to any other language or framework.
From here learning the hidden states is straight-forward, as we simply need to provide the 

% How account for different sample rates? How include this as part of observation state?


% If we have no transfer learning
% To achieve our goal we opted to build a model of 3 parts.
% \\\tempnote{Would be 2 parts if can get transfer learning to work, but having issues. Will just leave as preprocessing of the image prior to feeding into model if unable to get it sorted.}\\
% The first stage is to identify if a face is present in an image. For this we utilise a prebuilt CNN which returns a bounding box of the face, along with the points of the eyes, nose, and mouth corners\cite{yu2022yunet}. %This is executed with OpenCV
% The bounding box and the landmarks, along with the average of the IMU data since the last image, are then passed into a neural network which aims to predict how the head and phone are moving through the 6 Degrees of freedom.
% If no bounding box or landmarks are found for a given image, we provide the previous bounding box and landmarks. Ideally we would not provide anything, however the network expects to 
% \tempnote{is input going to be padded with zeros for first frames / last frames, or require certain number of frames before attempting classification?}\\
% The second model is 2 models which will be trained to predict the direction of movement in each of the 6 DoF for the head and phone (the head model will also take the landmarks and bounding box as input).
% It will output as a 2d one-hot encoded array, each row being the Degree of Freedom, the column being the direction (0 = stationary, -1 = negative, +1 = positive).
% The output of the 2 can then be fed into a HMM trained to predict the gesture performed based on the derived motion. (possibly an RNN if easier)

% What models have been used for cascading (facial landmark and YuNet CNNs)

% Tools, issues

\subsubsection{Training}\nl
% Breakdown of samples (train, validation, test), and count
Prior to training the first half of proposed model, we wanted to increase the amount of effective data we have for training we performed some fps scaling of the data. 
This involved iterating through our collected data and only extracting frames if they \textit{would} have been available at a lower sample rate. For images and the MoCap data we simply took the last available data for the current timestamp, but for the accelerations we calculated the average based on the time elapsed, as using just the last value would not be representative of the acceleration of the phone during the period.
In deployment of the model this would require that the accelerations are averaged in between images being captured, but this should provide greater resolution on how the phone is moving.\\
\tempnote{link to repo with python notebook for doing the fps scaling}\\
% \tempnote{Additionally generated the required classification output for the motion encoders.}
% 1029 recordings of gestures.
% K-Fold validation

% Hyper-params (quantisation, image size (if transfer learning?), fps scales)
% Image input sizes


\subsubsection{Model Evaluation}
% Confusion Matrices
% compare models
\tempnote{Still pending training of the models, encountered issues with Tensorflow, now moving to pytorch}
% \subsection{Model Deployment}
% % How the models will be used (Android App using TFLite (and OpenCV?))

